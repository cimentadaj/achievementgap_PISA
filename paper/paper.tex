


% TODO: Talk about how the achievement gap has been widening in Malaysia and South Korea (or Japan?)
% Is efficiency related to opportunity here? Is efficiency what explains the inequality of opportunity?
% Note how many of these studies haven't really concentrated on who is getting better or worse: top or bottom?
% Talk more about how countries with high social mobility has been linked to smallest achievement gaps
% Talk about Durpiez and Dumay and how there's not relationship between inequality income - inequality achievement in contrast to reardon and anna who find some relationship.

\section{Research questions and hypothesis}

The overarching aim of this paper is to study the evolution of the high/low SES achievement gap in the past 15 years for all PISA participating countries and search for a likely explanation for the evolution.

\noindent
We develop the subquestions and their corresponding hypothesis separately in more detail.

\begin{enumerate}

\item The seminal work of \citet{reardon2011} suggests that achievement gaps change, and they do so much quicker than we though after recording a SES gap increase of about 40\% in 30 years. \citet{reardon_portilla} stress that they also found a significant decrease in only 15 years of data, showing how important it is to study the changes in the achievement gap. First, we will concentrate on the evolution of the gap only for 15 year olds (which as we've seen before, there are reasons to think that specific age-groups have seen changes in the achievement gap), which will serve as a comparison to the single year-country snapshot of \citet{anna2016} and the evolution of the kindergarten gap in \citet{reardon_portilla}. Secondly, we will compare the percentage change at which the gap widened/narrowed from the first to the last year available. This will give us a general idea of the overall change over time, and will allow us to compare our estimates to the actual literature. \footnote{Although no study has performed this age-specific achievement gap for comparable tests over such a long time. Our results will serve as comparison for other studies that use age-specific groups, such as 4th graders using PIRLS.}. We posit no specific hypothesis for this question given that it's purely explanatory.

\begin{itemize}
\item Research question:
\begin{enumerate}

\item Which countries are experiencing changes in the achievement gap?

\end{enumerate}
\end{itemize}

\item The widening/narrowing of the achievement gap has a source, which has been often studied to be related to everything from educational spending, income inequality, time allocation to students and preschool enrollment. The literature has concentrated very narrowly on whether the gap is increasing because the top performers are getting ahead, because the bottom performers are falling behind or because both are changing at the same time. We shall pay particular attention to identifying the rate at which the top/bottom groups are evolving over time. Following that strategy will put us closer to understanding the source of the evolution of the achievement gap. 

\begin{itemize}
\item Research question:
\begin{enumerate}

\item Is the gap originating because the top is gaining ground, the top is falling behind or because of a dynamic interaction between the two?

\item Hypothesis 1: The theoretical argument in favor of tracking posits that in countries where there is a high degree of tracking we should expect the top and bottom to be evolving at a similar rate given that tracking is thought to maximize the learning experience of both groups. We expect to find the opposite, i.e for countries with a highly tracked curriculum, we should find more inequality and children from low-SES groups at a greater disadvantage. Conversely, in countries with low tracking we should expect for the achievement gap to be narrowing.

\end{enumerate}
\end{itemize}

\item Our study itself is composed of six cross-sectional surveys, which limits the design to single country snapshots, but given that most countries participated in all six surveys we can model the evolution of the achievement gap over time. We want to test whether several dimensions of the tracking setup and vocational enrollment explain the changes in the gaps, something already discussed in \citet{werfhorst_mijs} and \citet{hanushek_woesmann_tracking} but not empirically tested before in an international context.

\begin{itemize}
\item Research question:
\begin{enumerate}

\item Are tracking and vocational enrollment related to the evolution of the gap?

\item Hypothesis 2: Building on the previous hypothesis, tracking should play an important role on the evolution of the achievement gap. We hypothesize that the degree of tracking of an educational system is tightly related to changes in the gap, and the more tracking, the more inequality. Moreover, the more vocational tracking, the less inequality considering that it gives short term returns in terms of labor market opportunities.

\end{enumerate}
\end{itemize}

\item Lastly, we want to investigate whether better performing countries have lower levels of inequality than other countries. \citet{werfhorst_mijs} emphasize that there is empirical evidence that suggests this. This pattern is not so obvious, however. For example, countries with high levels of tracking could maximize student performance, specially the high SES students, raising their overall performance and thus raising the national performance score. But if the bottom performers are not gaining at the same rate, then the achievement gap will inevitable grow resulting in a high performing countries with a widening achievement gap.

\begin{itemize}
\item Research question:
\begin{enumerate}

\item Are achievement and inequality correlated? and if so, has the relationship weakened/strengthen over time?

\item Hypothesis 3: Building on \citet{werfhorst_mijs}, then there should be a negative relationship between the average country performance and their level of inequality. Countries with high performance should have less inequality and countries with low performance should have greater inequality

\end{enumerate}
\end{itemize}

\end{enumerate}

\section{Methods}
\subsection{Data}

\begin{knitrout}
\definecolor{shadecolor}{rgb}{0.969, 0.969, 0.969}\color{fgcolor}\begin{kframe}
\begin{alltt}
\hlstd{country_rows} \hlkwb{<-}
  \hlkwd{map_dbl}\hlstd{(adapted_year_data, nrow)} \hlopt
  \hlkwd{format}\hlstd{(}\hlkwc{big.mark} \hlstd{=} \hlstr{","}\hlstd{)}
\end{alltt}


{\ttfamily\noindent\bfseries\color{errorcolor}{\#\# Error in map\_dbl(adapted\_year\_data, nrow) \%>\% format(big.mark = "{},"{}): could not find function "{}\%>\%"{}}}\end{kframe}
\end{knitrout}

























































































